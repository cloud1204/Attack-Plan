敵人共有 $n$ 個據點,由 $m$ 條雙向道路互相連接。第 $i$ 條道路連接的是第 $u_i$ 個和第 $v_i$ 個據點,且該道路的長度為 $w_i$,代表要通過該道路需要花費 $w_i$ 秒。保證任意兩個據點之間都是直接或間接的連接的。

現在,你組織了一個 $n$ 個人的部隊,想要佔領敵人全部的據點。一開始(第 $0$ 秒時),所有人都在第 $s$ 個據點。

每個據點都有兩個數值代表其進攻的難度;當兵隊抵達第 $i$ 個據點的時候,根據抵達的小隊人數是否大於一,執行下列兩種操作之一:

\begin{enumerate}
  \item 如果小隊人數至少有兩個人或更多,則派遣部隊的其中一個人開始進攻該據點,其餘人繼續移動。留下來的那個人將會花 $a_i$ 秒(從他開始進攻的時間算起)成功佔領據點。

  \item 如果小隊人數只有一人,則直接開始進攻該據點。已知他將會花 $b_i$ 秒(從他開始進攻的時間算起)成功佔領該據點。
\end{enumerate}

因為被小隊拋下一個人進攻會很不甘情願,所以對於每個據點都滿足 $a_i \geq b_i$ 的性質。

在兵隊移動的時候,你可以自由指揮。兵分多路的時候,每個路拆分出來的小隊人數將由你自由決定(其中幾條路也可以 $0$ 個人走)。需要特別注意的規則是:你無法讓一個小隊移動到有人正在進攻或是已經進攻完成的據點,也不能讓兩個小隊同時抵達同一個據點。

在你的精心指揮下,最快需要幾秒才可以將所有據點都佔領?
